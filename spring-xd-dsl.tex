\section{DSL}
\label{sec:DSL}
Streams and jobs are defined in Spring XD via a DSL\cite{dsl}. The DSL allows developers
to quickly prototype and experiment with different stream and job definitions at
runtime without requiring a build tool or IDE. The DSL is provided by the end user
via the Spring XD shell or web UI.

\subsection {Pipes and Filters}
A simple stream definition consists of a sequence of modules. At minimum it requires a
source and sink module, with 0 or more processor modules in between. As a simple example
consider the collection of data from an HTTP source writing to a file sink:

\begin{lstlisting}
http | file
\end{lstlisting}

A stream that involves some processing:

\begin{lstlisting}
http | filter | transform | file
\end{lstlisting}

The modules in a stream definition are connected together using the pipe symbol |.

\subsection{Module Parameters}
Each module may take parameters. The parameters supported by a module are defined by
the module implementation. For example, the HTTP source module exposes a port setting
which allows the data ingestion port to be changed from the default value.

\begin{lstlisting}
http --port=1337
\end{lstlisting}

\subsection{Labels}

Labels provide a means to alias or group modules. Labels are a name followed
by the colon symbol :. When used as an alias a label can provide a more descriptive
name for a particular configuration of a module.

\begin{lstlisting}
http | obfuscator: transform
 --expression=
 payload.replaceAll('password','*')
 | file
\end{lstlisting}

Labels are required for disambiguating when multiple modules of the same name are used:

\begin{lstlisting}
mystream = http | uppercaser: transform
 --expression=payload.toUpperCase()
 | exclaimer: transform 
 --expression=payload+'!'
 | file
\end{lstlisting}

