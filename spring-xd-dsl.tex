\section{DSL Support}
Spring XD provides a DSL for defining a stream. Over time the DSL is likely to evolve significantly as it gains the ability to define more and more sophisticated streams as well as the steps of a batch job.
\subsection {Pipes and filters}
A simple linear stream consists of a sequence of modules. Typically an Input Source, (optional) Processing Steps, and an Output Sink. As a simple example consider the collection of data from an HTTP Source writing to a File Sink. Using the DSL the stream description is:

\verb;http | file;

A stream that involves some processing:

\verb;http | filter | transform | file;

The modules in a stream definition are connected together using the pipe symbol |.

\subsection{Module parameters}
Each module may take parameters. The parameters supported by a module are defined by the module implementation. As an example the http source module exposes port setting which allows the data ingestion port to be changed from the default value.

\verb;http --port=1337;

It is only necessary to quote parameter values if they contain spaces or the | character. Here the transform processor module is being passed a SpEL expression that will be applied to any data it encounters:

\verb;transform --expression=;\\*
 \verb;'new StringBuilder(payload).reverse()';\\* 

If the parameter value needs to embed a single quote, use two single quotes:

// Query is: Select * from /Sample where name='uid'

\verb;scan --query='Select * from /Sample;\\*
\verb; where name=''uid''';\\*

\subsection{Labels}

Labels provide a means to alias or group modules. Labels are simply a name followed by a : When used as an alias a label can provide a more descriptive name for a particular configuration of a module and possibly something easier to refer to in other streams.

\verb;http | obfuscator: transform;\\* 
\verb; --expression=payload.replaceAll('password','*');\\*
\verb; | file;\\*

Labels are especially useful (and required) for disambiguating when multiple modules of the same name are used:

\verb;mystream = http | uppercaser: transform ;\\* 
\verb; --expression=payload.toUpperCase();\\*
\verb; | exclaimer: transform --expression=payload+'!';\\*
\verb; | file;\\*



