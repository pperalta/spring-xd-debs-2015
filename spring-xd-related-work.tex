\section{Related Work}
This section compares and contrasts Spring XD to similar projects.

\subsection{Spark}
Spark\cite{spark} is a general purpose framework for large scale data processing.
In comparison with Hadoop's disk based MapReduce programming model, Spark's
in-memory primitives yields performance improvements.

The following features differentiate Spring XD and Spark Streaming.

\begin{itemize*}
\item Ability to microbatch based on event count via Reactor and RxJava APIs.
\item Ability to create data pipelines to process one event at a time.
\item Flexibility to specify hosts to dictate the location of data computations.
\end{itemize*}

The following features differentiate Spring XD and Spark Batch processing.

\begin{itemize*}
\item Provides REST-API and lifecycle management for Spark jobs.
\item Extensible to integrate with other batch systems.
\end{itemize*}

Recognizing the strengths of distributed data computations with Spark, Spring XD
supports integration with Spark applications such as Spark Streaming, MLLib, and
SparkSQL. Users familiar with Spark may implement the computation logic using
Spark APIs in Java or Scala, and leave the orchestration to Spring XD.
The Spring XD Spark module acts as the driver while the computation defined
by the streaming application is executed in the Spark cluster. Driver failure
is automatically handled by Spring XD -- the admin server will re-deploy a
module from a failed container to another eligible container.

\subsection{Spring XD and Storm}
Apache Storm\cite{storm} is a distributed computation system for real time stream
processing.

The following features differentiate Spring XD and Storm.

\begin{itemize*}
\item Spring XD provides an interactive shell while Storm requires programming
to an API to create data pipelines.
\item Use of `taps' in Spring XD to allows the creation of stream pipelines
in isolation without having to disrupt existing pipelines.
\item Loosely coupled modules in Spring XD are responsible for ingestion, analytics,
data processing, machine learning or data export. Modules can be individually managed
and dynamically scaled. Additionally modules may be co-located to minimize
serialization and network usage.
\item Building upon the functional stream processing model, users have the option
to choose from Reactor\cite{reactor}, Spark Streaming or RxJava APIs to build
complex data centric applications.
\end{itemize*}

Storm and Spring XD support many common data sources and middleware.
For example, the reading and writing of data payloads from Apache Kafka
is supported in both in Storm and Spring XD. A Bolt in Storm is analogous to a
source in Spring XD, and spouts in Storm are similar to processors and sinks
in Spring XD. As stream processing frameworks, Storm and Spring XD can be used for
similar use-cases.

\subsection{Flume}
Apache Flume\cite{flume} is a distributed system for collecting, aggregating and 
moving large data sets.

The following features differentiate Spring XD and Flume.

\begin{itemize*}
\item Spring XD uses an interactive shell and DSL for stream creation,
while Flume uses property (key/value pair) files.
\item Administration and monitoring via the admin UI.
\item Granular controls to manifest batch job and step execution to create
complex data driven workflows.
\item Flexibility through a deployment manifest to declaratively configure data
partitioning strategy to route data to a specific consumer instance in the cluster.
\end{itemize*}

Flume offers HBase, Solr, and ElasticSearch sinks along with encryption support
for Avro sources, which we are planning to address in our future releases.

\subsection{Oozie}
Oozie\cite{oozie} is a workflow scheduler engine to manage Hadoop \cite{hadoop} 
workloads such as MapReduce or Pig jobs.

The following features differentiate Spring XD and Oozie.

\begin{itemize*}
\item Building upon Spring Batch, a JSR standardization (JSR-352) of batch
workload data processing, Spring XD inherits workflow scheduling and execution
functionalities.
\item Provides out of the box batch jobs that support plain files, JDBC, HDFS,
FTP, MongoDB, Spark and Sqoop.
\item Ability to scale jobs without having to bring down the runtime.
\item Provides bi-directionality between real-time streaming and batch 
workflows to accommodate complex data processing use cases.
\item Ability to create, and launch jobs from the admin UI.
\item Ability to view historical snapshots of job executions from the admin UI.
\end{itemize*}

Oozie offers HCatalog integration, which we are planning to address in our 
future releases.

\subsection{Sqoop}
Apache Sqoop\cite{sqoop} assists with data transmission between Hadoop and relational
databases.

The following features differentiate Spring XD and Sqoop.

\begin{itemize*}
\item Ability to orchestrate Pig, Hive, HBase, MapReduce or other batch systems.
\item Flexibility to extend batch workflow infrastructure to write custom tasklets.
\item High level configuration DSL to create, deploy and destroy batch workflows.
\item Flexibility to operationalize custom data pipelines through REST.
\item Unified functional programming support to build reactive-style data pipelines.
\end{itemize*}

Sqoop offers data validation, data merge, incremental data imports, and HCatalog
integration among others. Recognizing the importance of these enterprise features,
Spring XD provides an out of the box Sqoop job to take advantage of these features.
