\section{Related Work}
This section compares and contrasts Spring XD to similar projects.

\subsection{Spring XD and Spark}
Spark\cite{spark} is a general-purpose framework for large scale data processing.
In comparison with Hadoop's disk based MapReduce programming model, Spark's 
in-memory primitives provide immediate performance improvements.

The following features differentiate Spring XD and Spark Streaming.

\begin{itemize*}
\item Readily available integration adapters for data movements from various 
data sources in to Hadoop and others.
\item Ability to microbatch based on event count via Reactor and RxJava APIs.
\item Ability to create data pipelines to process one event at a time.
\item Flexibility to specify hosts to dictate the location of data computations.
\end{itemize*}

The following features differentiate Spring XD and Spark Batch processing.

\begin{itemize*}
\item Provides REST-API and lifecycle management for Spark jobs.
\item Extensible to integrate with other Batch systems.
\end{itemize*}

Recognizing the strengths of distributed data computations with Spark, Spring XD 
supports integration with Spark applications such as Spark Streaming, MLLib, and 
SparkSQL. Users familiar with Spark may implement the computation logic using 
Spark APIs in Java or Scala, and leave the orchestration to Spring XD. 
Spring XD also adds value by restarting the Spark Streaming driver to recover 
from fault scenarios.

\subsection{Spring XD and Storm}
Apache Storm\cite{storm} is a distributed computation system for real time stream 
processing.

The following features differentiate Spring XD and Storm.

\begin{itemize*}
\item Spring XD provides interactive Shell as opposed to Storm's API model to
create data pipelines.
\item Use of `taps' in Spring XD to build stream pipelines in isolation
without having to disrupt existing pipelines.
\item Loosely coupled `modules' in Spring XD are responsible for ingestion, analytics, 
data processing, machine learning or data export. Modules can be individually managed 
and dynamically scaled.
\item The notion of `composite modules' (unit\-of\-work) and colocation 
capabilities to fine-tune performance characteristics. 
\item Building upon the functional stream processing model, users have the option 
to choose from Reactor\cite{reactor}, Spark Streaming or RxJava APIs, to build 
complex data centric applications.
\end{itemize*}

Storm and Spring XD supports many common data sources and middleware's. 
For example, you can read and write data payloads from Apache Kafka topics both 
in Storm and Spring XD. Bolts/Spouts in Storm is analogous to Source/Processor/Sink 
in Spring XD. As stream processing frameworks, Storm and Spring XD can be used for 
similar use-cases.

\subsection{Spring XD and Flume}
Apache Flume\cite{flume} is a distributed system for collecting, aggregating and 
moving large data sets. 

The following features differentiate Spring XD and Flume.

\begin{itemize*}
\item High-level DSL to build streams and jobs.
\item Ability to monitor data workflows either via DSL, Admin UI, or custom 
dashboards.
\item Administer data pipelines through Admin UI.
\item Granular controls to manifest batch job and step execution to create 
complex data driven workflows.
\item Flexibility through `Deployment Manifest' to declaratively configure data 
partitioning strategy to route data to a specific consumer instance in the cluster.
\end{itemize*}

Flume offers HBase, Solr, and ElasticSearch sinks along with encryption support 
for Avro sources, which we are planning to address in our future releases.

\subsection{Spring XD and Oozie}
Oozie\cite{oozie} is a workflow scheduler engine to manage Hadoop \cite{hadoop} 
workloads such as MapReduce or Pig jobs. 

The following features differentiate Spring XD and Oozie.

\begin{itemize*}
\item Building upon Spring Batch, a JSR standardization (JSR-352) of batch 
workload data processing, Spring XD inherits workflow scheduling and execution 
functionaliites.
\item Provides out of the box batch jobs such as file-to-jdbc, file-to-hdfs, 
ftp-to-hdfs, hdfs-to-jdbc, hdfs-to-mongo, jdbc-to-hdfs, spark-job, and sqoop-job.
\item Ability to scale jobs without having to bring down the runtime.
\item Provides bidirectionality between real-time streaming and batch 
workflows to accommodate complex data processing use cases.
\item Ability to create and launch workflow-jobs from Admin UI. 
\item Ability to view historical snapshots of job executions from Admin UI.
\end{itemize*}

Oozie offers HCatalog integration, which we are planning to address in our 
future releases.

\subsection{Spring XD and Sqoop}
Apache Sqoop\cite{sqoop} assists with data transmission between Hadoop and relational 
databases.

The following features differentiate Spring XD and Sqoop.

\begin{itemize*}
\item Ability to orchestrate Pig, Hive, HBase, MapReduce or other batch systems.
\item Flexibility to extend batch workflow infrastructure to write custom tasklets.
\item High level configuration DSL to create, deploy and destroy batch workflows.
\item Flexibility to operationalize custom data pipelines through REST-APIs.
\item Unified functional programming support to build reactive-style data pipelines.
\end{itemize*}

Sqoop offers data validation, data merge, incremental data imports, and HCatalog 
integration among others. Recognizing the importance of these enterprise features, 
Spring XD provides an out of the box Sqoop job to take advantage and orchestrate 
data movements.